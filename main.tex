\documentclass{article}
\usepackage{csquotes}
\usepackage[backend=biber]{biblatex}
\addbibresource{references.bib}
\usepackage{graphicx} % Required for inserting images
\usepackage{geometry}
\usepackage{amsmath}
\usepackage{amssymb}
\usepackage{booktabs}
\usepackage[spanish]{babel}
\usepackage{xcolor}
% beta 
% \pagecolor[rgb]{0,0,0} \color[rgb]{1,1,1}
\DeclareUnicodeCharacter{202F}{\,}

% URL breaking configuration (biblatex already loads url package)
\def\UrlBreaks{\do\/\do-\do_\do.\do=\do?\do&}
\urlstyle{same}

% Better line breaking to handle overfull hbox
\sloppy

\title{Preliminar: TFG}
\author{Ignacio de Miguel}
\date{September 2025}

\begin{document}

\maketitle

\section{Preliminar, la paradoja de Simpson}
Supongamos una población de 700 personas en la que hay una nueva epidemia para la que tenemos que averiguar si una nueva medicación es apropiada.
Un estudio podría consistir en averiguar la tasa de curación en pacientes que toman la medicación y compararla con la tasa de curación de los pacientes no medicados.

Extraemos la siguiente tabla del libro de Pearl:

\begin{table}[ht]
    \centering
    \caption{Results of a study into a new drug, with gender being taken into account}
    \label{tab:drug-gender}
    \begin{tabular}{@{}lcc@{}}
    \toprule
    & \textbf{Drug} & \textbf{No drug} \\
    \midrule
    Men            & 81 out of 87 recovered (93\%)   & 234 out of 270 recovered (87\%) \\
    Women          & 192 out of 263 recovered (73\%) & 55 out of 80 recovered (69\%)   \\
    Combined data  & 273 out of 350 recovered (78\%) & 289 out of 350 recovered (83\%) \\
    \bottomrule
    \end{tabular}
\end{table}

Centrémonos en la última fila. Vaya, hay una tasa de mayor recuperación entre los pacientes que no toman la medicación frente a los pacientes medicados. Podemos entonces concluir que la medicación no es efectiva.

El problema se descubre cuando visualizamos los datos teniendo en cuenta el género de los pacientes medicados versus aquellos no medicados. Si separamos por género, ahora tenemos que la medicación es más efectiva tanto en el caso de hombres como mujeres.

Esto da lugar a la paradoja de Simpson: la medicación es efectiva y la medicación no es efectiva. Es más, considerando que \textbf{no} sabemos el género de la persona que queremos tratar, deberíamos \textbf{no darle} la medicación. Una vez esta persona nos desvela si es hombre o mujer, \textbf{ya deberíamos} darle la medicación, pues los datos indican que en cualquiera de los casos mejorará.

Absurdo. Si ayuda a hombres y mujeres, debería ayudar a cualquiera de los dos. El problema, según interpreto yo, es que los hombres están infrarrepresentados en "tomar la medicación", teniendo menos peso en el cómputo total. 

Se ve de forma más extrema con el siguiente ejemplo:

\begin{table}[ht]
    \centering
    \caption{Results of a study into a new drug, with gender being taken into account}
    \label{tab:drug-gender}
    \begin{tabular}{@{}lcc@{}}
    \toprule
    & \textbf{Drug} & \textbf{No drug} \\
    \midrule
    Men            & 1 out of 1 recovered (100\%)   & 234 out of 270 recovered (87\%) \\
    Women          & 192 out of 263 recovered (73\%) & 55 out of 80 recovered (69\%)   \\
    Combined data  & 193 out of 350 recovered (55\%) & 289 out of 350 recovered (83\%) \\
    \bottomrule
    \end{tabular}
\end{table}

También podríamos interpretar que el género es una variable confusora o \textit{confounding variable}. En este estudio, ser hombre se asocia con tener una mayor capacidad de recuperación, pero también con una menor probabilidad de tomar la medicación. Esto hace que la población que no toma la medicación (mayoritariamente hombres) se cure más, pues tiene una mayor capacidad de recuperación. Al separar por la variable confusora, se resuelve la confusión.

De hecho, la paradoja se resuelve suponiendo que averiguamos que el estrógeno actúa en contra de la capacidad curativa general, por lo que es menos probable que una mujer se cure. Además, se supone que es más probable que una mujer tome el medicamento, precisamente porque el efecto adverso de la enfermedad es mayor en la mujer.

Se me ocurre alguna pregunta, como por ejemplo: si suponemos que no sabemos que el estrógeno actúa en contra de la capacidad curativa general, ¿cómo sabemos que la separación entre hombres y mujeres no es arbitraria y por tanto inservible?
\end{document}