% !TEX encoding = UTF-8 Unicode
%\documentclass[a4paper,11pt,spanish, twoside, openany]{tfg-uam-matematicas_25-26} 
\documentclass[a4paper,11pt,spanish, twoside]{tfg-uam-matematicas_25-26} 

\usepackage[utf8]{inputenc}
\usepackage{amsfonts, amssymb, amsmath, amsthm}
\usepackage{graphicx}
\usepackage{color}
\usepackage{booktabs}
\usepackage{csquotes}
\usepackage[backend=biber,style=alphabetic]{biblatex}
\addbibresource{references.bib}


\newtheorem{teor}{Teorema}[chapter]
\newtheorem{lema}[teor]{Lema}
\newtheorem*{teorsin}{Teorema}


\theoremstyle{definition}
\newtheorem{defin}[teor]{Definición}

\title{Título del trabajo va aquí}
\author{Nombre y Apellidos del autor/a}
\curso{2025-2026}


%%%%%METADATOS: rellenar la info solicitada entre llaves
\usepackage{hyperref}
\hypersetup{
	pdfinfo={
            Title={ }, %Titulo del trabajo; ejemplo: Matematicas y desarrollo
            Author={ }, %Autor del trabajo. 
            Director1={ }, %Tutor1: en formato nombre.apellido, tal como aparece en la primera parte, antes de la arroba,  de su dirección de correo electrónico de la UAM; ejemplo: fernando.soria
            Director2={ }, %Tutor2: en formato nombre.apellido, tal como aparece en la primera parte, antes de la arroba,  de su dirección de correo electrónico de la UAM
            Ndirectores={ }, %Numero total de directores: 1 ó 2
            Tipo={TFG}, %no tocar
            Curso={2025-26}, %no tocar
            Palabrasclave={ },% Palabras clave del trabajo, separadas por comas y sin acentos ni espacios; ejemplo: morfismos, formas modulares, ecuaciones elipticas
				}
}
%%%%%%%%%%%%%%%%%%%%%%%%%%%%%%%

\begin{document}
\frontmatter
% Agradecimientos (Opcional)
\cleardoublepage
\null
\vfill
\begin{flushright}
    \textit{A todos los que me quieren.}
\end{flushright}
\vfill
\cleardoublepage
%%Fin de agradecimientos

%%Resumen, en español y en inglés (Obligatorios ambos)
\begin{abstract}[spanish]
Lorem ipsum dolor sit amet, consectetuer adipiscing elit. Aenean commodo ligula eget dolor. Aenean massa. Cum sociis natoque penatibus et magnis dis parturient montes, nascetur ridiculus mus. Donec quam felis, ultricies nec, pellentesque eu, pretium quis, sem. Nulla consequat massa quis enim. Donec pede justo, fringilla vel, aliquet nec, vulputate eget, arcu. In enim justo, rhoncus ut, imperdiet a, venenatis vitae, justo. Nullam dictum felis eu pede mollis pretium. Integer tincidunt. Cras dapibus. Vivamus elementum semper nisi. Aenean vulputate eleifend tellus. Aenean leo ligula, porttitor eu, consequat vitae, eleifend ac, enim. Aliquam lorem ante, dapibus in, viverra quis, feugiat a, tellus. Phasellus viverra nulla ut metus varius laoreet. Quisque rutrum. Aenean imperdiet. Etiam ultricies nisi vel augue. Curabitur ullamcorper ultricies nisi. Nam eget dui. Etiam rhoncus. Maecenas tempus, tellus eget condimentum rhoncus, sem quam semper libero, sit amet adipiscing sem neque sed ipsum. Nam quam nunc, blandit vel, luctus pulvinar, hendrerit id, lorem. Maecenas nec odio et ante tincidunt tempus. Donec vitae sapien ut libero venenatis faucibus. Nullam quis ante. Etiam sit amet orci eget eros faucibus tincidunt. Duis leo. Sed fringilla mauris sit amet nibh. Donec sodales sagittis magna. Sed consequat, leo eget bibendum sodales, augue velit cursus nunc,
\end{abstract}
\begin{abstract}[english]
Lorem ipsum dolor sit amet, consectetuer adipiscing elit. Aenean commodo ligula eget dolor. Aenean massa. Cum sociis natoque penatibus et magnis dis parturient montes, nascetur ridiculus mus. Donec quam felis, ultricies nec, pellentesque eu, pretium quis, sem. Nulla consequat massa quis enim. Donec pede justo, fringilla vel, aliquet nec, vulputate eget, arcu. In enim justo, rhoncus ut, imperdiet a, venenatis vitae, justo. Nullam dictum felis eu pede mollis pretium. Integer tincidunt. Cras dapibus. Vivamus elementum semper nisi. Aenean vulputate eleifend tellus. Aenean leo ligula, porttitor eu, consequat vitae, eleifend ac, enim. Aliquam lorem ante, dapibus in, viverra quis, feugiat a, tellus. Phasellus viverra nulla ut metus varius laoreet. Quisque rutrum. Aenean imperdiet. Etiam ultricies nisi vel augue. Curabitur ullamcorper ultricies nisi. Nam eget dui. Etiam rhoncus. Maecenas tempus, tellus eget condimentum rhoncus, sem quam semper libero, sit amet adipiscing sem neque sed ipsum. Nam quam nunc, blandit vel, luctus pulvinar, hendrerit id, lorem. Maecenas nec odio et ante tincidunt tempus. Donec vitae sapien ut libero \end{abstract}

\cleardoublepage

% Lista de símbolos (Opcional)
% Título con mismo estilo que \chapter
\begin{flushright}
\Huge\bf Lista de símbolos\\*[-.5\baselineskip]
\hrulefill
\end{flushright}


\begin{tabular}{ll}
$\in$ & Pertenece a \\
$\notin$ & No pertenece a \\
$\subseteq$ & Subconjunto de \\
$\subset$ & Subconjunto propio de \\
$\cup$ & Unión de conjuntos \\
$\cap$ & Intersección de conjuntos \\
$\emptyset$ & Conjunto vacío \\
$\mathbb{N}$ & Conjunto de números naturales \\
$\mathbb{Z}$ & Conjunto de números enteros \\
$\mathbb{Q}$ & Conjunto de números racionales \\
$\mathbb{R}$ & Conjunto de números reales \\
$\mathbb{C}$ & Conjunto de números complejos \\
$f: A \to B$ & Función de $A$ en $B$ \\
$\mathrm{Id}$ & Identidad \\
$\ker(f)$ & Núcleo de una aplicación \\
$\mathrm{Im}(f)$ & Imagen de una aplicación \\
$\|x\|$ & Norma de un vector \\
$\langle x,y \rangle$ & Producto interno \\
$|A|$ & Cardinal de $A$ \\
$\forall$ & Para todo \\
$\exists$ & Existe \\
$\Rightarrow$ & Implica \\
$\Leftrightarrow$ & Equivalencia (si y sólo si) \\
$\sum_{i=1}^n a_i$ & Suma finita \\
$\prod_{i=1}^n a_i$ & Producto finito \\
$\lim_{x \to a} f(x)$ & Límite de $f$ en $a$ \\
$f'(x)$ & Derivada de $f$ \\
$\int_a^b f(x)\,dx$ & Integral definida \\
$O(g(n))$ & Cota asintótica superior \\
$\cong$ & Isomorfismo \\
$\simeq$ & Equivalencia \\
$\approx$ & Aproximadamente igual \\
$\triangleq$ & Definido como \\
$\mathbb{P}(A)$ & Probabilidad del suceso $A$ \\
$\mathbb{E}[X]$ & Esperanza de $X$ \\
$\mathrm{Var}(X)$ & Varianza de $X$ \\
\end{tabular}
\cleardoublepage

%%%%%%%%%%%%%%%%%%%%%%%%%%%%%%%%%%%%%%%%%%%%%%%%%%%%%%%%%
%%%%%%%%%%%%%%%%%%%%%%%%%%%%%%%%%%%%%%%%%%%%%%%%%%%%%%%%%
%%%%%%%%%%%%%%%%%%%%%%%%%%%%%%%%%%%%%%%%%%%%%%%%%%%%%%%%%
%%%%%%%%%%%%%%%%%%%%%%%%%%%%%%%%%%%%%%%%%%%%%%%%%%%%%%%%%
\mainmatter

%%%A partir de aquí el cuerpo del trabajo  %%%%%%%

\chapter*{Introducción}  % Capítulo sin numeración
\addcontentsline{toc}{chapter}{Introducción}  % Agrega la introducción al índice
testing testing inasdfas
Lorem ipsum dolor sit amet, consectetuer adipiscing elit. Aenean commodo ligula eget dolor. Aenean massa. Cum sociis natoque penatibus et magnis dis parturient montes, nascetur ridiculus mus. Donec quam felis, ultricies nec, pellentesque eu, pretium quis, sem. Nulla consequat massa quis enim. Donec pede justo, fringilla vel, aliquet nec, vulputate eget, arcu. In enim justo, rhoncus ut, imperdiet a, venenatis vitae, justo. Nullam dictum felis eu pede mollis pretium.
\begin{itemize}\itemsep=0pt
\item
Integer tincidunt.

\item
Cras dapibus.
\item
Vivamus elementum semper nisi.
\item
Aenean vulputate eleifend tellus.
\end{itemize}

Aenean leo ligula, porttitor eu, consequat vitae, eleifend ac, enim. Aliquam lorem ante, dapibus in, viverra quis, feugiat a, tellus. Phasellus viverra nulla ut metus varius laoreet. Quisque rutrum. Aenean imperdiet. Etiam ultricies nisi vel augue. Curabitur ullamcorper ultricies nisi. Nam eget dui. Etiam rhoncus. Maecenas tempus, tellus eget condimentum rhoncus, sem quam semper libero, sit amet adipiscing sem neque sed ipsum. Nam quam nunc, blandit vel, luctus pulvinar, hendrerit id, lorem. Maecenas nec odio et ante tincidunt tempus. Donec vitae sapien ut libero venenatis faucibus. Nullam quis ante. Etiam sit amet orci eget eros faucibus tincidunt. Duis leo. Sed fringilla mauris sit amet nibh. Donec sodales sagittis magna. Sed consequat, leo eget bibendum sodales, augue velit cursus nunc. Véanse \cite{Abel} y \cite{S-W}.



\chapter{Introducción}\label{chap1}
Una de las motivaciones del análisis de datos es tratar de descubrir la existencia de relaciones causales entre variables aleatorias. Es común intentar utilizar la estadística para responder a preguntas como `¿funciona la medicina X para tratar Y?' o bien `¿causa el factor X un descenso/aumento de Y?'. Según Pearl~\cite{pearl2016causal}.



\section{Motivación}



\section{La paradoja de Simpson}
Supongamos una población de 700 personas en la que hay una nueva epidemia para la que tenemos que averiguar si una nueva medicación es apropiada.
Un estudio podría consistir en averiguar la tasa de curación en pacientes que toman la medicación y compararla con la tasa de curación de los pacientes no medicados.

\begin{table}[ht]
    \centering
    \caption{Results of a study into a new drug, with gender being taken into account}
    \label{tab:drug-gender}
    \begin{tabular}{@{}lcc@{}}
    \toprule
    & \textbf{Drug} & \textbf{No drug} \\
    \midrule
    Men            & 81 out of 87 recovered (93\%)   & 234 out of 270 recovered (87\%) \\
    Women          & 192 out of 263 recovered (73\%) & 55 out of 80 recovered (69\%)   \\
    Combined data  & 273 out of 350 recovered (78\%) & 289 out of 350 recovered (83\%) \\
    \bottomrule
    \end{tabular}
\end{table}



Centrémonos en la última fila. Vaya, hay una tasa de mayor recuperación entre los pacientes que no toman la medicación frente a los pacientes medicados. Podemos entonces concluir que la medicación no es efectiva.

El problema se descubre cuando visualizamos los datos teniendo en cuenta el género de los pacientes medicados versus aquellos no medicados. Si separamos por género, ahora tenemos que la medicación es más efectiva tanto en el caso de hombres como mujeres.

Esto da lugar a la paradoja de Simpson: la medicación es efectiva y la medicación no es efectiva. Es más, considerando que \textbf{no} sabemos el género de la persona que queremos tratar, deberíamos \textbf{no darle} la medicación. Una vez esta persona nos desvela si es hombre o mujer, \textbf{ya deberíamos} darle la medicación, pues los datos indican que en cualquiera de los casos mejorará.

Absurdo. Si ayuda a hombres y mujeres, debería ayudar a cualquiera de los dos. El problema, según interpreto yo, es que los hombres están infrarrepresentados en "tomar la medicación", teniendo menos peso en el cómputo total. 

Se ve de forma más extrema con el ejemplo visible en la tabla~\ref{tab:table-extreme}

\begin{table}[ht]
    \centering
    \caption{Results of a study into a new drug, with gender being taken into account}
    \label{tab:drug-gender}
    \begin{tabular}{@{}lcc@{}}
    \toprule
    & \textbf{Drug} & \textbf{No drug} \\
    \midrule
    Men            & 1 out of 1 recovered (100\%)   & 234 out of 270 recovered (87\%) \\
    Women          & 192 out of 263 recovered (73\%) & 55 out of 80 recovered (69\%)   \\
    Combined data  & 193 out of 264 recovered (73\%) & 289 out of 350 recovered (83\%) \\
    \bottomrule
    \end{tabular} \label{tab:table-extreme}
\end{table}

También podríamos interpretar que el género es una variable confusora o \textit{confounding variable}. En este estudio, ser hombre se asocia con tener una mayor capacidad de recuperación, pero también con una menor probabilidad de tomar la medicación. Esto hace que la población que no toma la medicación (mayoritariamente hombres) se cure más, pues tiene una mayor capacidad de recuperación. Al separar por la variable confusora, se resuelve la confusión. 
De hecho, la paradoja se resuelve suponiendo que averiguamos que el estrógeno actúa en contra de la capacidad curativa general, por lo que es menos probable que una mujer se cure. Además, se supone que es más probable que una mujer tome el medicamento, precisamente porque el efecto adverso de la enfermedad es mayor en la mujer.


\section{Resultados preliminares}

Lorem ipsum dolor sit amet, Teorema \ref{teor1}, consectetuer adipiscing elit. Aenean commodo ligula eget dolor. Aenean massa. Cum sociis natoque penatibus et magnis dis parturient montes, nascetur ridiculus mus. Donec quam felis, ultricies nec, pellentesque eu, pretium quis, sem. Nulla consequat massa quis enim. Donec pede justo, fringilla vel, aliquet nec, vulputate eget, arcu. In enim justo, rhoncus ut, imperdiet a, venenatis vitae, justo. Nullam dictum felis eu pede mollis pretium. Integer tincidunt. Cras dapibus. Vivamus elementum semper nisi. Aenean vulputate eleifend tellus. Aenean leo ligula, porttitor eu, consequat vitae, eleifend ac, enim. Aliquam lorem ante, dapibus in, viverra quis, feugiat a, tellus. Phasellus viverra nulla ut metus varius laoreet. Quisque rutrum. Aenean imperdiet. Etiam ultricies nisi vel augue. Curabitur ullamcorper ultricies nisi. Nam eget dui. Etiam rhoncus. Maecenas tempus, tellus eget condimentum rhoncus, sem quam semper libero, sit amet adipiscing sem neque sed ipsum. Nam quam nunc, blandit vel, luctus pulvinar, hendrerit id, lorem. Maecenas nec odio et ante tincidunt tempus. Donec vitae sapien ut libero venenatis faucibus. Nullam quis ante. Etiam sit amet orci eget eros faucibus tincidunt. Duis leo. Sed fringilla mauris sit amet nibh. Donec sodales sagittis magna. Sed consequat, leo eget bibendum sodales, augue velit cursus nunc,
\begin{align}\label{eq4}
&e^{i\pi }+1=0,
\\
&2e^{i\pi }+2=0.\label{eq5}
\end{align}

Lorem ipsum dolor sit amet, consectetuer adipiscing elit. Aenean commodo ligula eget dolor.
\begin{align}\nonumber
0&=e^{i\pi }+1=e^{i\pi }+1=e^{i\pi }+1=e^{i\pi }+\sum_{n=1}^\infty \frac{1}{2^n}
\\
&=-1+\sum_{n=1}^\infty \frac{1}{2^n}=-1+1=0.\label{eq6}
\end{align}
et
\begin{equation}\label{eq7}
\begin{aligned}
e^{i\pi }+1=0,
\\
e^{i\pi }+1=0.
\end{aligned}
\end{equation}


Lorem ipsum dolor sit amet, consectetuer adipiscing elit. Aenean commodo ligula eget dolor.
\begin{align*}
e^{i\pi }+1&=0,
\\
e^{i\pi }+1&=0.
\end{align*}
Aenean massa: 
\begin{equation}
\left\{
\begin{array}{l}
e^{i\pi }+1=0,
\\
e^{i\pi }+1=0.
\end{array}
\right.
\end{equation}


Cum sociis natoque penatibus et magnis dis parturient montes, nascetur ridiculus mus. Donec quam felis, ultricies nec, pellentesque eu, pretium quis, sem. Nulla consequat massa quis enim. Donec pede justo, fringilla vel, aliquet nec, vulputate eget, arcu. In enim justo, rhoncus ut, imperdiet a, venenatis vitae, justo. Nullam dictum felis eu pede mollis pretium. Integer tincidunt. Cras dapibus. Vivamus elementum semper nisi. Aenean vulputate eleifend tellus. Aenean leo ligula, porttitor eu, consequat vitae, eleifend ac, enim. Aliquam lorem ante, dapibus in, viverra quis, feugiat a, tellus. Phasellus viverra nulla ut metus varius laoreet. Quisque rutrum. Aenean imperdiet. Etiam ultricies nisi vel augue. Curabitur ullamcorper ultricies nisi. Nam eget dui. Etiam rhoncus. Maecenas tempus, tellus eget condimentum rhoncus, sem quam semper libero, sit amet adipiscing sem neque sed ipsum. Nam quam nunc, blandit vel, luctus pulvinar, hendrerit id, lorem. Maecenas nec odio et ante tincidunt tempus. Donec vitae sapien ut libero venenatis faucibus. Nullam quis ante. Etiam sit amet orci eget eros faucibus tincidunt. Duis leo. Sed fringilla mauris sit amet nibh. Donec sodales sagittis magna. Sed consequat, leo eget bibendum sodales, augue velit cursus nunc,

\clearpage

\chapter{El segundo capítulo}\label{chap2}


\section{Uno más}

%%%%%%%%%%%%%%%%%%%%%%%%%%%%%%%%%%%%%%%%%%%%%%%%%%%%%%%%%
%%%%%%%%%%%%%%%%%%%%%%%%%%%%%%%%%%%%%%%%%%%%%%%%%%%%%%%%%
%%%% Final del cuerpo %%%%%%%%%%%%%%%%%%%%%%%%%%%%%%%%%%%
%%%%%%%%%%%%%%%%%%%%%%%%%%%%%%%%%%%%%%%%%%%%%%%%%%%%%%%%%
%%%%%%%%%%%%%%%%%%%%%%%%%%%%%%%%%%%%%%%%%%%%%%%%%%%%%%%%%
% Apéndices (Opcionales)
\appendix 
\chapter{Resultados auxiliares}
En este apéndice recopilamos algunas definiciones y lemas auxiliares que se utilizan a lo largo del texto. Aunque la mayoría de estos resultados son bien conocidos, los incluimos aquí para mayor claridad y comodidad del lector. Comenzamos con una desigualdad elemental que será empleada repetidamente en las demostraciones de los teoremas principales.

\begin{lema}[Desigualdad básica]
Sean $a,b \in \mathbb{R}$ con $b \neq 0$. Entonces
\[
\left| \frac{a}{b} \right| \leq \frac{|a|}{|b|}.
\]
Además, la igualdad se cumple si y solo si $ab \geq 0$.
\end{lema}

\begin{proof}
La prueba se deduce directamente de la propiedad multiplicativa del valor absoluto, es decir, $|ab| = |a||b|$ para todo $a,b \in \mathbb{R}$. Dividiendo ambos lados por $|b|^2$ obtenemos la desigualdad. La caracterización de la igualdad es inmediata.
\end{proof}


\chapter{Otro apéndice}

\printbibliography

\cleardoublepage


\end{document}
