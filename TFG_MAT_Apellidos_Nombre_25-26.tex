% !TEX encoding = UTF-8 Unicode
%\documentclass[a4paper,11pt,spanish, twoside, openany]{tfg-uam-matematicas_25-26} 
\documentclass[a4paper,11pt,spanish, twoside]{tfg-uam-matematicas_25-26} 

\usepackage[utf8]{inputenc}
\usepackage{amsfonts, amssymb, amsmath, amsthm}
\usepackage{graphicx}
\usepackage{color}
\usepackage{booktabs}
\usepackage{csquotes}

\usepackage[backend=biber,style=alphabetic]{biblatex}
\addbibresource{references.bib}

\usepackage{tikz}
\usetikzlibrary{decorations.markings, positioning}



\newtheorem{teor}{Teorema}[chapter]
\newtheorem{lema}[teor]{Lema}
\newtheorem*{teorsin}{Teorema}


\theoremstyle{definition}
\newtheorem{defin}[teor]{Definición}

\title{Causalidad y Estadística}
\author{Ignacio Ildefonso de Miguel Ruano}
\curso{2025-2026}


%%%%%METADATOS: rellenar la info solicitada entre llaves
\usepackage{hyperref}
\hypersetup{
	pdfinfo={
            Title={ }, %Titulo del trabajo; ejemplo: Matematicas y desarrollo
            Author={ }, %Autor del trabajo. 
            Director1={ }, %Tutor1: en formato nombre.apellido, tal como aparece en la primera parte, antes de la arroba,  de su dirección de correo electrónico de la UAM; ejemplo: fernando.soria
            Director2={ }, %Tutor2: en formato nombre.apellido, tal como aparece en la primera parte, antes de la arroba,  de su dirección de correo electrónico de la UAM
            Ndirectores={ }, %Numero total de directores: 1 ó 2
            Tipo={TFG}, %no tocar
            Curso={2025-26}, %no tocar
            Palabrasclave={ },% Palabras clave del trabajo, separadas por comas y sin acentos ni espacios; ejemplo: morfismos, formas modulares, ecuaciones elipticas
				}
}
%%%%%%%%%%%%%%%%%%%%%%%%%%%%%%%

\begin{document}
\frontmatter
% Agradecimientos (Opcional)
\cleardoublepage
\null
\vfill
\begin{flushright}
    \textit{A todos los que me quieren.}
\end{flushright}
\vfill
\cleardoublepage
%%Fin de agradecimientos

%%Resumen, en español y en inglés (Obligatorios ambos)
\begin{abstract}[spanish]
Lorem ipsum dolor sit amet, consectetuer adipiscing elit. Aenean commodo ligula eget dolor. Aenean massa. Cum sociis natoque penatibus et magnis dis parturient montes, nascetur ridiculus mus. Donec quam felis, ultricies nec, pellentesque eu, pretium quis, sem. Nulla consequat massa quis enim. Donec pede justo, fringilla vel, aliquet nec, vulputate eget, arcu. In enim justo, rhoncus ut, imperdiet a, venenatis vitae, justo. Nullam dictum felis eu pede mollis pretium. Integer tincidunt. Cras dapibus. Vivamus elementum semper nisi. Aenean vulputate eleifend tellus. Aenean leo ligula, porttitor eu, consequat vitae, eleifend ac, enim. Aliquam lorem ante, dapibus in, viverra quis, feugiat a, tellus. Phasellus viverra nulla ut metus varius laoreet. Quisque rutrum. Aenean imperdiet. Etiam ultricies nisi vel augue. Curabitur ullamcorper ultricies nisi. Nam eget dui. Etiam rhoncus. Maecenas tempus, tellus eget condimentum rhoncus, sem quam semper libero, sit amet adipiscing sem neque sed ipsum. Nam quam nunc, blandit vel, luctus pulvinar, hendrerit id, lorem. Maecenas nec odio et ante tincidunt tempus. Donec vitae sapien ut libero venenatis faucibus. Nullam quis ante. Etiam sit amet orci eget eros faucibus tincidunt. Duis leo. Sed fringilla mauris sit amet nibh. Donec sodales sagittis magna. Sed consequat, leo eget bibendum sodales, augue velit cursus nunc,
\end{abstract}
\begin{abstract}[english]
Lorem ipsum dolor sit amet, consectetuer adipiscing elit. Aenean commodo ligula eget dolor. Aenean massa. Cum sociis natoque penatibus et magnis dis parturient montes, nascetur ridiculus mus. Donec quam felis, ultricies nec, pellentesque eu, pretium quis, sem. Nulla consequat massa quis enim. Donec pede justo, fringilla vel, aliquet nec, vulputate eget, arcu. In enim justo, rhoncus ut, imperdiet a, venenatis vitae, justo. Nullam dictum felis eu pede mollis pretium. Integer tincidunt. Cras dapibus. Vivamus elementum semper nisi. Aenean vulputate eleifend tellus. Aenean leo ligula, porttitor eu, consequat vitae, eleifend ac, enim. Aliquam lorem ante, dapibus in, viverra quis, feugiat a, tellus. Phasellus viverra nulla ut metus varius laoreet. Quisque rutrum. Aenean imperdiet. Etiam ultricies nisi vel augue. Curabitur ullamcorper ultricies nisi. Nam eget dui. Etiam rhoncus. Maecenas tempus, tellus eget condimentum rhoncus, sem quam semper libero, sit amet adipiscing sem neque sed ipsum. Nam quam nunc, blandit vel, luctus pulvinar, hendrerit id, lorem. Maecenas nec odio et ante tincidunt tempus. Donec vitae sapien ut libero \end{abstract}

\cleardoublepage

% Lista de símbolos (Opcional)
% Título con mismo estilo que \chapter
\begin{flushright}
\Huge\bf Lista de símbolos\\*[-.5\baselineskip]
\hrulefill
\end{flushright}


\begin{tabular}{ll}
$\in$ & Pertenece a \\
$\notin$ & No pertenece a \\
$\subseteq$ & Subconjunto de \\
$\subset$ & Subconjunto propio de \\
$\cup$ & Unión de conjuntos \\
$\cap$ & Intersección de conjuntos \\
$\emptyset$ & Conjunto vacío \\
$\mathbb{N}$ & Conjunto de números naturales \\
$\mathbb{Z}$ & Conjunto de números enteros \\
$\mathbb{Q}$ & Conjunto de números racionales \\
$\mathbb{R}$ & Conjunto de números reales \\
$\mathbb{C}$ & Conjunto de números complejos \\
$f: A \to B$ & Función de $A$ en $B$ \\
$\mathrm{Id}$ & Identidad \\
$\ker(f)$ & Núcleo de una aplicación \\
$\mathrm{Im}(f)$ & Imagen de una aplicación \\
$\|x\|$ & Norma de un vector \\
$\langle x,y \rangle$ & Producto interno \\
$|A|$ & Cardinal de $A$ \\
$\forall$ & Para todo \\
$\exists$ & Existe \\
$\Rightarrow$ & Implica \\
$\Leftrightarrow$ & Equivalencia (si y sólo si) \\
$\sum_{i=1}^n a_i$ & Suma finita \\
$\prod_{i=1}^n a_i$ & Producto finito \\
$\lim_{x \to a} f(x)$ & Límite de $f$ en $a$ \\
$f'(x)$ & Derivada de $f$ \\
$\int_a^b f(x)\,dx$ & Integral definida \\
$O(g(n))$ & Cota asintótica superior \\
$\cong$ & Isomorfismo \\
$\simeq$ & Equivalencia \\
$\approx$ & Aproximadamente igual \\
$\triangleq$ & Definido como \\
$\mathbb{P}(A)$ & Probabilidad del suceso $A$ \\
$\mathbb{E}[X]$ & Esperanza de $X$ \\
$\mathrm{Var}(X)$ & Varianza de $X$ \\
\end{tabular}
\cleardoublepage

%%%%%%%%%%%%%%%%%%%%%%%%%%%%%%%%%%%%%%%%%%%%%%%%%%%%%%%%%
%%%%%%%%%%%%%%%%%%%%%%%%%%%%%%%%%%%%%%%%%%%%%%%%%%%%%%%%%
%%%%%%%%%%%%%%%%%%%%%%%%%%%%%%%%%%%%%%%%%%%%%%%%%%%%%%%%%
%%%%%%%%%%%%%%%%%%%%%%%%%%%%%%%%%%%%%%%%%%%%%%%%%%%%%%%%%
\mainmatter

%%%A partir de aquí el cuerpo del trabajo  %%%%%%%

\chapter*{Introducción}  % Capítulo sin numeración
\addcontentsline{toc}{chapter}{Introducción}  % Agrega la introducción al índice
testing testing inasdfas
Lorem ipsum dolor sit amet, consectetuer adipiscing elit. Aenean commodo ligula eget dolor. Aenean massa. Cum sociis natoque penatibus et magnis dis parturient montes, nascetur ridiculus mus. Donec quam felis, ultricies nec, pellentesque eu, pretium quis, sem. Nulla consequat massa quis enim. Donec pede justo, fringilla vel, aliquet nec, vulputate eget, arcu. In enim justo, rhoncus ut, imperdiet a, venenatis vitae, justo. Nullam dictum felis eu pede mollis pretium.
\begin{itemize}\itemsep=0pt
\item
Integer tincidunt.

\item
Cras dapibus.
\item
Vivamus elementum semper nisi.
\item
Aenean vulputate eleifend tellus.
\end{itemize}

Aenean leo ligula, porttitor eu, consequat vitae, eleifend ac, enim. Aliquam lorem ante, dapibus in, viverra quis, feugiat a, tellus. Phasellus viverra nulla ut metus varius laoreet. Quisque rutrum. Aenean imperdiet. Etiam ultricies nisi vel augue. Curabitur ullamcorper ultricies nisi. Nam eget dui. Etiam rhoncus. Maecenas tempus, tellus eget condimentum rhoncus, sem quam semper libero, sit amet adipiscing sem neque sed ipsum. Nam quam nunc, blandit vel, luctus pulvinar, hendrerit id, lorem. Maecenas nec odio et ante tincidunt tempus. Donec vitae sapien ut libero venenatis faucibus. Nullam quis ante. Etiam sit amet orci eget eros faucibus tincidunt. Duis leo. Sed fringilla mauris sit amet nibh. Donec sodales sagittis magna. Sed consequat, leo eget bibendum sodales, augue velit cursus nunc. Véanse \cite{Abel} y \cite{S-W}.



\chapter{Introducción}\label{chap1}



\section{El problema de correlación vs. causalidad}
La estadística estándar utiliza conceptos asociacionales entre variables (correlación, dependencia, regresión\ldots). Estos conceptos se definen en base a una distribución conjunta de variables, $P(x_1, x_2, \ldots)$. El problema es que las relaciones causales no son definibles únicamente con una distribución de variables conjunta. Supongamos que $X$ e $Y$ son variables aleatorias de Bernoulli, cualesquiera que sean sus interpretaciones. Supongamos que observamos los datos de~\ref{tab:correlacion_simple}.

\begin{table}[h]
    \centering
    \begin{tabular}{lccc}
        \toprule
         & \multicolumn{2}{c}{\textbf{$Y$}} & \\
        \cmidrule(lr){2-3}
        \textbf{$X$} & \textbf{Sí ($Y=1$)} & \textbf{No ($Y=0$)} & \textbf{Total} \\
        \midrule
        \textbf{Sí ($X=1$)} & \textbf{380} & 20 & 400 \\
        \textbf{No ($X=0$)} & 50 & \textbf{550} & 600 \\
        \midrule
        \textbf{Total} & 430 & 570 & 1000 \\
        \bottomrule
    \end{tabular}
    \caption{Tabla de observaciones de $X$ e $Y$}
    \label{tab:correlacion_simple}
\end{table}

Un cálculo rápido de las probabilidades condicionales nos lleva a que $P(X=1|Y=1) \approx 0.88$. Entonces, lo que podríamos decir es que, \textbf{dado que hemos observado $Y = 1$, es muy probable que hayamos observado $X=1$}. Es entonces cuando debemos resistirnos a afirmar que \textbf{$Y=1$ causa $X=1$}. Sin un `cuento semántico' que dé interpretaciones a las variables, las relaciones inferibles entre las variables no pueden ser causales \cite{DBLP:journals/corr/HuangV12}.

Supongamos que interpretamos $X$ como `el suelo está mojado' e $Y$ como `la gente lleva paraguas'. Ahora se vuelve más claro por qué no podemos afirmar que $X = 1$ cause $Y = 1$. Incluso existiendo una relación causal entre las variables, afirmaciones de este tipo pueden no ser ciertas: supongamos ahora que interpretamos $Y$ como `ha llovido'. Sabemos de sobra que $Y = 1$ causaría $X = 1$, pero si partimos de las probabilidades condicionales, sería indistinguible decir esto de decir que $X = 1$ causa $Y = 1$. Es decir, afirmar que llover causa que el suelo esté mojado porque las variables están correlacionadas sería tan válido como afirmar que el que el suelo esté mojado cause que llueva por la misma razón.

En definitiva, \textbf{$N$ variables no están causalmente relacionadas por estar correlacionadas}. Si volvemos a nuestra interpretación original de las variables $X$ e $Y$, sabemos de sobra que existe una variable $Z$ que podemos interpretar como `llueve', que \textbf{directamente} causa que el suelo esté mojado y que la gente lleve paraguas (y, en general, para cualquier par de variables $X$ e $Y$, si existe una correlación entre $X$ e $Y$ pero no hay relación causal, debe existir una variable $Z$ que explique esa correlación entre ambas~\cite{Reichenbach1956}). El propio razonamiento ya nos lleva a deducir que \textbf{no es suficiente} con los razonamientos estadísticos estándar. Debemos elaborar un cuento causal, similar a lo que se vería en la Figura~\ref{fig:dag_causal}.

\begin{figure}[h]
\centering
\begin{tikzpicture}[
    node distance=2.5cm,
    decoration={markings, mark=at position 1 with {\arrow{stealth}}},
    every node/.style={circle, draw, minimum size=1cm, thick, fill=white}
]

% Nodos
\node (Z) {$Z$};
\node[below left=of Z] (X) {$X$};
\node[below right=of Z] (Y) {$Y$};

% Aristas
\draw[thick, postaction={decorate}] (Z) -- (X);
\draw[thick, postaction={decorate}] (Z) -- (Y);

\end{tikzpicture}
\caption{Grafo que representa las relaciones causales entre $X$, $Y$ y $Z$}
\label{fig:dag_causal}
\end{figure}

La diferencia clave entre la estadística tradicional y el estudio de la causalidad es la intervención. Una manera de saber si hay una relación causal entre dos variables es la de responder a la pregunta: \textbf{¿cambiaría la variable $X$ si modifico la variable $Y$?}, mientras que, originalmente, en estadística preguntamos \textbf{¿hemos observado que un cambio en $X$ está asociado a un cambio en la variable $Y$?}
\section{La paradoja de Simpson}
Supongamos una población en la que se ha descubierto una nueva epidemia. Rápidamente, los doctores responsables del bienestar de dicha población se ponen manos a la obra para intentar descubrir un remedio. Un estudio preliminar observacional sobre una medicación $M$ arroja los datos que podemos ver en la Tabla~\ref{tab:drug-no-gender}.

\begin{table}[ht]
    \centering
    \caption{Resultados de aplicar la medicación a la población descrita}
    \label{tab:drug-no-gender}
    \begin{tabular}{@{}lcc@{}}
    \toprule
    & \textbf{Tomaron la medicina} & \textbf{No tomaron la medicina} \\
    \midrule
    Datos de curación  & 273 de 350 se curaron (78\%) & 289 de 350 se curaron (83\%) \\
    \bottomrule
    \end{tabular}
\end{table}

Considerando estos datos, lo más sensato sería retirar de inmediato la medicación, pues causa una tasa de curación menor. Sin embargo, nuevos datos revelan que \textbf{las mujeres son más vulnerables a la enfermedad que los hombres}. Se decide entonces separar por hombres y mujeres la tabla, para ver qué efectos tiene en ambos sexos por separado. La tabla resultante es la~\ref{tab:drug-gender}

\begin{table}[ht]
    \centering
    \caption{Resultados de aplicar la medicación a la población descrita, esta vez separando por sexo}
    \label{tab:drug-gender}
    \begin{tabular}{@{}lcc@{}}
    \toprule
    & \textbf{Tomaron la medicina} & \textbf{No tomaron la medicina} \\
    \midrule
    Hombres            & 81 de 87 se curaron (93\%)   & 234 de 270 se curaron (87\%) \\
    Mujeres          & 192 de 263 se curaron (73\%) & 55 de 80 se curaron (69\%)   \\
    Datos de curación  & 273 de 350 se curaron (78\%) & 289 de 350 se curaron (83\%) \\
    \bottomrule
    \end{tabular}
\end{table}

Si llamamos $H$ a `ser hombre' y $C$ a `curarse', tenemos que, como antes:

\begin{equation*}
    P(C|M) < P(C|\neg M)
\end{equation*}

Pero, si condicionamos por el sexo, las desigualdades se invierten:
\begin{align*}
    P(C\mid M, H) &> P(C\mid \neg M, H) \\
    P(C\mid M, \neg H) &> P(C\mid \neg M, \neg H)
\end{align*}

Es decir, al no observar el sexo del paciente, la probabilidad de curarse dado que se toma la medicación es menor que la de curarse sin tomarse nada, pero \textbf{una vez hemos observado el sexo del paciente}, concluimos que sí debe tomar la medicación. Carece de sentido lógico. 

Si nos centramos en los números de la tabla separada por sexo vemos que los grupos no son homogéneos. Hay más mujeres que tomaron la medicina que hombres. De igual manera, más hombres no tomaron la medicina que mujeres.

Volvemos a la afirmación anterior: las mujeres son más vulnerables que los hombres. Como más mujeres tomaron la medicina, la medicina tuvo que `enfrentarse' a casos más `difíciles' de resolver más frecuentemente. Es decir, en el caso de tomarse la medicina, esta se enfrentó más veces a casos más difíciles de curar. En el caso de no tomarla, la `no-medicina' se enfrentó mayoritariamente a casos fáciles de curar.

En este caso en particular, supongamos que se realiza una investigación en profundidad del criterio que utilizaron los médicos para recetar o no el remedio. Descubrimos que, ante la incertidumbre de saber si realmente surtiría un efecto positivo, lo emplearon con mayor tendencia en casos de mayor riesgo, en este caso los pacientes de sexo femenino.

Esto causó los datos que vemos en la tabla, donde el remedio se enfrentaba en su mayoría a casos de mujeres. Si ahora intentamos construir un grafo similar a la Figura~\ref{fig:dag_causal}, podemos establecer varias relaciones causales:

\begin{enumerate}
    \item La curación de un paciente depende de si se le da la medicina o no.
    \item La curación de un paciente depende de su sexo.
    \item El que se recete el remedio a un paciente depende de su sexo.
\end{enumerate}

Por tanto, el grafo que podríamos construir sería similar al de la Figura~\ref{fig:simpson-dag}

\begin{figure}[ht]
    \centering
    \begin{tikzpicture}[>=stealth, node distance=2cm]
        % Nodos
        \node[draw, circle] (G) at (0, 2) {Sexo ($Z$)};
        \node[draw, circle] (T) at (-2, 0) {Tratamiento ($X$)};
        \node[draw, circle] (R) at (2, 0) {Recuperación ($Y$)};

        % Flechas (Causalidad)
        \draw[->, thick] (G) -- (T) node[midway, above left] {sesgo};
        \draw[->, thick] (G) -- (R) node[midway, above right] {pronóstico};
        \draw[->, thick] (T) -- (R) node[midway, above] {?};
    \end{tikzpicture}
    \caption{Diagrama Causal (DAG) de la Paradoja de Simpson. La variable $Z$ (Sexo) es una variable de confusión que afecta tanto a la asignación del tratamiento como al resultado.}
    \label{fig:simpson-dag}
\end{figure}


\section{Resultados preliminares}

Lorem ipsum dolor sit amet, Teorema \ref{teor1}, consectetuer adipiscing elit. Aenean commodo ligula eget dolor. Aenean massa. Cum sociis natoque penatibus et magnis dis parturient montes, nascetur ridiculus mus. Donec quam felis, ultricies nec, pellentesque eu, pretium quis, sem. Nulla consequat massa quis enim. Donec pede justo, fringilla vel, aliquet nec, vulputate eget, arcu. In enim justo, rhoncus ut, imperdiet a, venenatis vitae, justo. Nullam dictum felis eu pede mollis pretium. Integer tincidunt. Cras dapibus. Vivamus elementum semper nisi. Aenean vulputate eleifend tellus. Aenean leo ligula, porttitor eu, consequat vitae, eleifend ac, enim. Aliquam lorem ante, dapibus in, viverra quis, feugiat a, tellus. Phasellus viverra nulla ut metus varius laoreet. Quisque rutrum. Aenean imperdiet. Etiam ultricies nisi vel augue. Curabitur ullamcorper ultricies nisi. Nam eget dui. Etiam rhoncus. Maecenas tempus, tellus eget condimentum rhoncus, sem quam semper libero, sit amet adipiscing sem neque sed ipsum. Nam quam nunc, blandit vel, luctus pulvinar, hendrerit id, lorem. Maecenas nec odio et ante tincidunt tempus. Donec vitae sapien ut libero venenatis faucibus. Nullam quis ante. Etiam sit amet orci eget eros faucibus tincidunt. Duis leo. Sed fringilla mauris sit amet nibh. Donec sodales sagittis magna. Sed consequat, leo eget bibendum sodales, augue velit cursus nunc,
\begin{align}\label{eq4}
&e^{i\pi }+1=0,
\\
&2e^{i\pi }+2=0.\label{eq5}
\end{align}

Lorem ipsum dolor sit amet, consectetuer adipiscing elit. Aenean commodo ligula eget dolor.
\begin{align}\nonumber
0&=e^{i\pi }+1=e^{i\pi }+1=e^{i\pi }+1=e^{i\pi }+\sum_{n=1}^\infty \frac{1}{2^n}
\\
&=-1+\sum_{n=1}^\infty \frac{1}{2^n}=-1+1=0.\label{eq6}
\end{align}
et
\begin{equation}\label{eq7}
\begin{aligned}
e^{i\pi }+1=0,
\\
e^{i\pi }+1=0.
\end{aligned}
\end{equation}


Lorem ipsum dolor sit amet, consectetuer adipiscing elit. Aenean commodo ligula eget dolor.
\begin{align*}
e^{i\pi }+1&=0,
\\
e^{i\pi }+1&=0.
\end{align*}
Aenean massa: 
\begin{equation}
\left\{
\begin{array}{l}
e^{i\pi }+1=0,
\\
e^{i\pi }+1=0.
\end{array}
\right.
\end{equation}


\clearpage

\chapter{El segundo capítulo}\label{chap2}


\section{Uno más}

%%%%%%%%%%%%%%%%%%%%%%%%%%%%%%%%%%%%%%%%%%%%%%%%%%%%%%%%%
%%%%%%%%%%%%%%%%%%%%%%%%%%%%%%%%%%%%%%%%%%%%%%%%%%%%%%%%%
%%%% Final del cuerpo %%%%%%%%%%%%%%%%%%%%%%%%%%%%%%%%%%%
%%%%%%%%%%%%%%%%%%%%%%%%%%%%%%%%%%%%%%%%%%%%%%%%%%%%%%%%%
%%%%%%%%%%%%%%%%%%%%%%%%%%%%%%%%%%%%%%%%%%%%%%%%%%%%%%%%%
% Apéndices (Opcionales)
\appendix 
\chapter{Resultados auxiliares}
En este apéndice recopilamos algunas definiciones y lemas auxiliares que se utilizan a lo largo del texto. Aunque la mayoría de estos resultados son bien conocidos, los incluimos aquí para mayor claridad y comodidad del lector. Comenzamos con una desigualdad elemental que será empleada repetidamente en las demostraciones de los teoremas principales.

\begin{lema}[Desigualdad básica]
Sean $a,b \in \mathbb{R}$ con $b \neq 0$. Entonces
\[
\left| \frac{a}{b} \right| \leq \frac{|a|}{|b|}.
\]
Además, la igualdad se cumple si y solo si $ab \geq 0$.
\end{lema}

\begin{proof}
La prueba se deduce directamente de la propiedad multiplicativa del valor absoluto, es decir, $|ab| = |a||b|$ para todo $a,b \in \mathbb{R}$. Dividiendo ambos lados por $|b|^2$ obtenemos la desigualdad. La caracterización de la igualdad es inmediata.
\end{proof}


\chapter{Otro apéndice}

\printbibliography

\cleardoublepage


\end{document}
